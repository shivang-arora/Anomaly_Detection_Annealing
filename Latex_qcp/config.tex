% +------------------------------------------------------------------------------+ %
% | EDIT THESE SETTINGS ACCORDING TO YOUR THESIS                                 | %
% +------------------------------------------------------------------------------+ %

\newcommand{\practical}{Quantum Computing Programmierung} % comment for master seminar
%\newcommand{\seminar}{Vertiefte Themen in Mobilen und Verteilten Systemen} % uncomment for master seminar

\newcommand{\authorA}{Magdalena Benkard}
\newcommand{\authorB}{Thomas Holger}
\newcommand{\authorC}{Wanja Sajko}
\newcommand{\authorD}{Dani\"{e}lle Schuman}
\newcommand{\supervisor}{Mallory Miller}

\newcommand{\thesistitle}{Anomalieerkennung mittels einer Quantum Boltzmann Machine}

\newcommand{\thesisabstract}{% \/ put your thesis abstract below \/
Diese Arbeit bearbeitet das Problem der Anomalieerkennung mit Hilfe einer Quantum Boltzmann Machine. 
Hierbei handelt es sich um einen Ansatz, der auf dem maschinellem Erlernen einer Wahrscheinlichkeitsverteilung von Daten basiert. 
Zur Annäherung an diese wird ein hybrides Verfahren verwendet, bei dem ein Quantum Annealer zum Sampeln eingesetzt wird.
Zur Klassifizierung der Anomalien werden die Energiewerte der Datenpunkte verglichen.
In der Evaluierung wurden verschiedene Quantum Computer betrachtet und mit einem klassischen Ansatz verglichen.

}% <-- mind this closing brace!

\newcommand{\thesisauthorship}{% \/ describe who wrote what below \/
Magdalena Benkard hat das Kapitel~\ref{sec:intro} und die Abschnitte~\ref{subsec:basics-qc} und~\ref{subsec:basics-ml} verfasst. Wanja Sajko hat die Abschnitte~\ref{subsec:basics-ad} und \ref{subsec:rbm} verfasst. Daniëlle Schuman hat das Kapitel~\ref{sec:related}, den Abschnitt~\ref{subsec:qbm} und den Abschnitt~\ref{subsec:anomaly-detec-bm} verfasst. Thomas Holger hat das Kapitel~\ref{sec:evaluation} verfasst. Das Kapitel~\ref{sec:conclusion} und das Abstract haben alle Autoren gemeinsam verfasst.
}% <-- mind this closing brace!


\selectlanguage{ngerman} %options: english, ngerman